\chapter{Introduction}
Anaphora, or referring backwards \citep{mitkov1999}, is a common linguistic tool for improving the efficiency of expressions. As in \eqref{mary}
\begin{align}
 \mbox{\emph{Mary dropped the cup. It shattered loudly.}} \label{mary}
\end{align}
the pronoun points to the cup clearly, thus exempting us from the labor of repetition at only a trivial distance. In daily usage of languages, we are constantly faced with the task of pronoun resolution. However, pronouns in some cases do not provide adequate information for identifying the intended referent. From time to time, we may also encounter ambiguous anaphora, whether be they literary devices or simply bad writings. Nevertheless, successful language production and comprehension requires rapid interpretation of co-references, which raises the questions of what factors determine our choices of antecedents for pronouns. The problem anaphora resolution has been an active area of research across many disciplines \citep{sayed2003}. However, resolving pronouns properly, whether for a human or a machine could be a difficult task since the referent could be at various syntax positions, and be distracted by semantic elements within or outside the same sentence. Furthermore, things can get more complicated when a discourse involves several parties or non-native speakers. 

In order to link the pronoun to its appropriate antecedent, people usually insert intentionally or subconsciously the referent into a salient position. The increase of the salience of a specific item in a sentence or discourse can be realized via various means, including topicalization, focus and focus-sensitive particles \citep{patterson2017}, which make up different information structures. \citet{徐晓东2014} distinguish linguistic (top-down) focus from non-linguistic (bottom-up) ones. The former includes cleft structure, which we will be focusing on in this thesis and wh-question etc.; and the latter mostly comprises of tonal changes, especially pitch accent. In practice, the syntactic aspect of focus is highly correlated with its phonological patterns. From the point of view of generative grammar, phonology and semantics cannot exchange information directly, so that syntactic mechanism including transformations, prosodic information, or focus in particular, is passed between semantics and phonology \citep{beaver2008}. Also unlike topicalization, which provides the theme of a sentence or discourse, namely what is being talked about, focus usually introduces “new, non-derivable, or contrastive information” \citep{halliday1967}, and is concerned about what is being said about. It should be noted that although the two notions serve different linguistic purposes, they are not contradictory to each other, so that focus might also be at topic position and vice versa.

Cleft is a primary implementation of focus in many languages. A cleft sentence is a complex sentence (the combination of a matrix clause and a dependent clause) that has a meaning that could be expressed by a simple sentence also known as the canonical sentence. Clefts typically put a particular constituent into focus. In spoken language, this focusing is often accompanied by a special intonation. In English, a cleft sentence can be constructed as follows:
\begin{align}
    \mbox{\emph{it}} + \mbox{\emph{conjugated form of to be}} + \left[X_F\right] + \mbox{\emph{subordinate clause}}
\end{align}
where \emph{it} is a cleft pronoun and $X$ is usually a noun phrase (although it can also be a prepositional phrase, and in some cases an adjectival or adverbial phrase). The focus is on X, or else on the subordinate clause or some element of it. For example, in \eqref{smith} and \eqref{joseph},
\begin{align}
    &\mbox{\emph{It's Mr. Smith (whom) we're looking for.}}\label{smith}\\
    &\mbox{\emph{It was from Joseph that she heard the news.}}\label{joseph}
\end{align}
\emph{Mr. Smith} and \emph{from Joseph} are focused. Furthermore, cleft structure usually induces an obligatory intonation break, so we may put a nuclear pitch accent on the focused constituents, which results in the interactions between phonology and semantics we mentioned earlier. Similarly, clefts can be achieved by \emph{c'est} (it is) structure in French:
\begin{align}
    &\mbox{\emph{C'est Jacque que je cherche.} (It's Jacque whom I'm looking for.)}\label{jacque}\\
    &\mbox{\emph{C'est à Paris que j'habite.} (It's in Paris where I live.)}\label{paris}
\end{align}
As can be seen from the examples above, focus can be either the subject or object of a sentence. If pronouns are introduced in later discourses, the focus can either be their antecedents or not. In addition, other factors such as topic, verb semantics, intra- and inter-sentential differences and even second language influence may come to play a role in the interpretation of pronouns. In this regard, the effectiveness of focus should be questioned in certain context.

The present study will thus explore the impact of focus effect on pronoun resolution of non-native English and French speakers, aiming to tackle three main problems:
\begin{itemize}
    \item how focus is formed syntactically in English and French;
    \item to what extent does focus increase the salience of the referent;
    \item and whether or not non-native English and French speakers respond to cleft structures in the same way as native speakers?
\end{itemize}