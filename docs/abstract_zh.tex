%%% 中文摘要
\clearpage
\thispagestyle{plain}
\phantomsection
\addcontentsline{toc}{chapter}{摘\quad 要}

\centerline{\zihao{3}\heiti 摘\quad 要}

\linespread{1.4}\zihao{-4} \bigskip

本文探讨了英语和法语非母语者的语言处理过程中,焦点结构和代词回指之间的关系。首先我们从语法和语用角度,梳理了关于焦点作用和代词回指机制的研究现状。接着通过设计自定步速阅读实验,我们发现以分裂句为形式的焦点结构不一定能提升语言片段的显著性,并使其更可能成为代词回指的对象,这与前人的研究结果是相符的。在我们的测试中,法语主语位置的焦点和英语宾语位置的焦点反应时间都较短,但两种语言的主语焦点都导致了更长的回指反应时间。另外,我们还发现焦点和回指对象的一致性并不能提升焦点位置的显著性。因此本文认为,英语和法语中是否存在主语或宾语回指偏向的问题,比当前已有研究结论更加复杂。
\bigskip

\noindent{\zihao{4}\heiti 关键词:}
焦点作用, 代词回指, 自定步速阅读, 英语, 法语
